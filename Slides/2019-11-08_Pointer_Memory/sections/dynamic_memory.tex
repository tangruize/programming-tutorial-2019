\section{动态内存}\label{sec:动态内存}

\begin{frame}[fragile]{动态内存}
    \begin{itemize}[<+- | alert@+>]
        \item 前面写的程序使用固定大小的数组, 对于输入规模变化的情况怎么办?
        \item 需要动态分配内存空间!
        \item 本节主要介绍四个函数: \texttt{malloc}, \texttt{calloc}, \texttt{realloc}, \texttt{free}
    \end{itemize}
\end{frame}

\begin{frame}[fragile]{栈区(Stack)和堆区(Heap)}
    \begin{itemize}[<+- | alert@+>]
        \item 一个函数调用产生一个栈帧, 栈帧中存储的局部变量的大小一般在编译时确定, 且生存期与函数调用期间内
        \item 堆区存储了动态分配的内存, 该内存区域中的变量生存期不随函数调用变化
        \item 更多讨论: \href{https://stackoverflow.com/questions/79923/what-and-where-are-the-stack-and-heap}{stackoverflow.com}
    \end{itemize}
\end{frame}

\begin{frame}[fragile]{malloc}
    \scriptsize\begin{lstlisting}[language=c]
#include <stdlib.h>
void *malloc(size_t size);
  // Returns pointer to allocated memory on success, or NULL on error
    \end{lstlisting}
    \normalsize
    \begin{itemize}[<+- | alert@+>]
        \item \texttt{malloc} 接受参数size, 表示需分配空间大小的\textbf{字节数}
        \item 返回一个void *指针指向从堆区分配的\textbf{未初始化}内存区域, 或NULL表示错误
        \item 如果size为0, Linux的行为是返回一个指针指向一块小的内存区域而不是NULL
        \item 示例代码:
        \scriptsize\begin{verbatim}
    int * p = (int *)malloc(1024 * sizeof(int));
        \end{verbatim}
    \end{itemize}
\end{frame}

\begin{frame}[fragile]{calloc}
    \scriptsize\begin{lstlisting}[language=c]
#include <stdlib.h>
void *calloc(size_t numitems , size_t size);
  // Returns pointer to allocated memory on success, or NULL on error
    \end{lstlisting}
    \normalsize
    \begin{itemize}[<+- | alert@+>]
        \item 注意到malloc只有一个size参数, 且不会初始化分配的内存
        \item calloc用于从堆区分配一个数组, 有numitems个元素, 每个元素大小为size
        \item calloc将分配的内存区域初始化为0
        \item calloc返回值与malloc相同
        \item 示例代码:
        \scriptsize\begin{verbatim}
    struct { /* Some field definitions */ } myStruct;
    struct myStruct *p;
    p = calloc(1000, sizeof(struct myStruct));
        \end{verbatim}
    \end{itemize}
\end{frame}

\begin{frame}[fragile]{realloc}
    \scriptsize\begin{lstlisting}[language=c]
#include <stdlib.h>
void *realloc(void * ptr , size_t size);
  // Returns pointer to allocated memory on success, or NULL on error
    \end{lstlisting}
    \normalsize
    \begin{itemize}[<+- | alert@+>]
        \item 从函数名不难猜出, 该函数用于对已经分配的内存区域调整大小
        \item 参数 ptr 是之前调用 malloc, calloc 或 realloc 的返回值, 该指针指向内存区域大小将被调整为size字节
        \item 返回值与malloc相同
        \item 如果调大size, realloc会尝试合并相邻内存区域, 如果不能合并则会重新分配内存并将原来的数据复制到新的区域, 增加的内存区域不会被初始化
        \item 示例代码:
        \scriptsize\begin{verbatim}
    nptr = realloc(ptr, newsize);
    if (nptr == NULL) {
        /* Handle error */
    } else {  /* realloc() succeeded */
        ptr = nptr;
    }
        \end{verbatim}
    \end{itemize}
\end{frame}

\begin{frame}[fragile]{free}
    \scriptsize\begin{lstlisting}[language=c]
#include <stdlib.h>
void free(void * ptr);
    \end{lstlisting}
    \normalsize
    \begin{itemize}[<+- | alert@+>]
        \item 用于归还malloc, calloc, realloc分配的内存ptr
        \item ptr如果为NULL, 则free什么也不做
        \item 使用free后尽量将ptr赋值为NULL, 避免重复free (会造成内存corruption)
        \item 内存安全的程序应该free掉不使用的内存, 避免内存泄漏
    \end{itemize}
\end{frame}

\begin{frame}[fragile]{与众不同的alloca}
    \scriptsize\begin{lstlisting}[language=c]
#include <alloca.h>  // or `malloc.h' in VS
void *alloca(size_t size);
  // Returns pointer to allocated block of memory
    \end{lstlisting}
    \normalsize
    \begin{itemize}[<+- | alert@+>]
        \item 前面的malloc, calloc, realloc都从堆区分配空间, 且必须用free将不使用的内存归还
        \item alloca参数和返回值与malloc类似, 但从栈区分配空间, 该区域的生命期与局部变量相同
        \item alloca分配的空间不需要也不能用free去释放
        \item alloca原理: 正在调用的函数栈帧位于栈顶, 栈帧还可以扩展, 因此不能这样用alloca:
        \scriptsize\begin{verbatim}
    func(x, alloca(size), z);  /* WRONG! */
        \end{verbatim}
        \item 正确的做法是:
        \scriptsize\begin{verbatim}
    void *y;
    y = alloca(size);
    func(x, y, z);
        \end{verbatim}
    \end{itemize}
\end{frame}

\begin{frame}[fragile]{alloca相比于堆区分配空间的优势}
    \begin{itemize}[<+- | alert@+>]
        \item 快: 在栈区分配空间只需要简单调大栈指针, 只需要几条机器指令
        \item 自动释放空间: 函数返回空间自动释放
        \item 有的函数调用不可避免造成动态内存区域泄露, 比如使用longjmp()函数进行非本地跳转, alloca自动释放空间更安全
    \end{itemize}
\end{frame}

\begin{frame}[fragile]{malloc是如何实现的?}
    \begin{center}
        请听下回分解
    \end{center}
\end{frame}

\begin{frame}[fragile]{malloc是如何实现的?}
    \begin{center}
        请听下回分解 \\
        \tiny 剧透一下是用链表实现的
    \end{center}
\end{frame}