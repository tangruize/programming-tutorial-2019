\section{数组}\label{sec:数组}

\begin{frame}[fragile]{数组定义}
    数组是一系列相同类型元素, 放置在一片连续的内存区域.
    每个元素都能单独被索引到

    定义方式: \texttt{type name [elements][elements];}

    \begin{itemize}[<+- | alert@+>]
        \item \texttt{type}: 数组类型
        \item \texttt{name}: 变量名
        \item \texttt{elements}: 元素个数
        \item 注意数组元素索引从0开始, 最大索引为 elements-1
    \end{itemize}
\end{frame}

\begin{frame}[fragile]{数组初始化}
    \begin{lstlisting}[language=c]
    int foo [5];
    int foo [5] = { 16, 2, 77, 40, 12071 };
    int bar [5] = { 10, 20, 30 };
    int baz [5] = { 0 };
    int foo [] = { 16, 2, 77, 40, 12071 };
    \end{lstlisting}

    \begin{itemize}[<+- | alert@+>]
        \item 第1行, 数组大小为5, 不对数组进行初始化, 可能是随机值, 不能在赋值前使用
        \item 第2行, 数组大小为5, 所有元素都提供了初值
        \item 第3行, 数组大小为5, 只提供了一部分初值, 其他的, 即 bar[3] 和 bar[4] 将被初始化为0
        \item 第4行, 数组大小为5, 所有元素都被初始化为0
        \item 第5行, 数组大小根据初始化提供的初值个数确定, 即5.
        如果提供了所有的初值或不需要显式指定元素个数时, 这种初始化比第2行好
    \end{itemize}
\end{frame}

\begin{frame}[fragile]{多维数组}
    多维数组可以当作是数组的数组, 比如 \texttt{int a[5][5];}

    但本质上仍然是一片连续的内存区域, \texttt{int a[5][5];} 等价于 \texttt{int b[25];}

    获取\texttt{a[3][4]} 等价于 \texttt{b[19]}, 如果两个维度不相同, 如果有三个维度, 你知道怎么算吗?
\end{frame}

\begin{frame}[fragile]{数组作为函数参数}
    \begin{itemize}[<+- | alert@+>]
        \item 函数参数格式: \texttt{type name[][elements][elements]}
        \item 即除了第一维大小可省略 , 其他维度都要给定大小
        \item 思考为什么要这样做, 与前面说的多维数组和一维数组等价表示有什么关系?
        \item 举例 \\
        函数声明为: \texttt{void procedure(int arg[][3], int n);}\\
        调用该函数: \texttt{procedure(a, 5);}
    \end{itemize}
\end{frame}